%Report on The Design of a White Lab Component Vending Machine
%Undergratuate Engineering Thesis
%Author: Baden David Morgan
%Supervisor: Justin Pead

%document formatting
\documentclass[a4paper,11pt]{article}

	% packages
\usepackage{graphicx}
\graphicspath{{C:/"Google Drive"/thesis/Document/Pictures/}}
\usepackage[margin=2cm]{geometry}
\usepackage{hyperref}

	% commands
\newcommand{\sign}[1]{%      
  \begin{tabular}[t]{@{}l@{}}
  \makebox[1.5in]{\dotfill}\\
  \strut#1\strut
  \end{tabular}%
  }
\newcommand{\Date}{%
  \begin{tabular}[t]{@{}p{1.5in}@{}}
  \\[-2ex]
  \strut Date: \dotfill\strut
  \end{tabular}%
}

\usepackage{titlesec}
%title formatting
\titleformat*{\section}{\center\Huge\bfseries}
%\titleformat*{\subsection}{\LARGE\bfseries}
\titleformat{\subsection}[block]{\LARGE\bfseries\hspace{1em}}{\thesubsection}{1em}{}
\titleformat{\subsubsection}[block]{\Large\bfseries\hspace{2em}}{\thesubsubsection}{1em}{}
\titleformat*{\paragraph}{\large\bfseries}
\titleformat*{\subparagraph}{\large\bfseries}

\begin{document}

	% title page
\begin{titlepage}
	\centering
	{\Huge\bfseries\underline{White Lab Component Vending Machine}\par}
	\vspace{1cm}
	{\huge Design and Build Report of a Component Vending Machine for the Undergraduates for White 	Lab\par}
	\vspace{1cm}
	\includegraphics[scale=0.2]{{UCTLogo}.jpg}
	\vfill
	Prepared by:\par
	Baden David Morgan\par
	MRGBAD001\par
	
	\vfill
	Prepared for:\par
	Mr. J. Pead\par
	Department of Electrical and Electronics Engineering\par
	University of Cape Town\par
	
	\vfill
	Submitted to the Department of Electrical Engineering at the University of Cape Town
	in partial fulfilment of the academic requirements for a Bachelor of Science degree in
	Mechatronic Engineering\par
	
	\vfill
	{\large October, 2016 \par}
	
	\vfill
	{\bfseries Key words:}
	this and that
\end{titlepage}

	% empty page
	\mbox{}
	\thispagestyle{empty}
	\newpage
	
	% plagiarism declaration
	\thispagestyle{empty}
	{\huge Plagiarism Declaration \par}
	\begin{enumerate}
  		\item I, \makebox[1in]{\hrulefill}, know that plagiarism is wrong. Plagiarism is to use another’s work and pretend
			that it is one’s own.
 		\item I, \makebox[1in]{\hrulefill}, have used the IEEE convention for citation and referencing. Each contribution to,
			and quotation in, this report from the work(s) of other people has been attributed,
			and has been cited and referenced
 		\item This report is my, \makebox[1in]{\hrulefill}, own work.
 		\item	I, \makebox[1in]{\hrulefill}, have not allowed, and will not allow, anyone to copy my work with the intention
			of passing it off as their own work or part thereof.
	\end{enumerate}
		\noindent\begin{tabular}{ll}
		\makebox[2.5in]{\hrulefill} & \makebox[2.5in]{\hrulefill}\\
		Full Name & Date\\[8ex]% adds space between the two sets of signatures
		\makebox[2.5in]{\hrulefill} \\
		Signature \\[8ex]% adds space between the two sets of signatures
	\end{tabular}
	\newpage
	
	% empty page
	\mbox{}
	\thispagestyle{empty}
	\newpage	
	
	%terms of reference
	\pagenumbering{roman}	
	
	{\centering\Huge\bfseries\underline{Terms of Reference}\par}
	\newpage
	
	%acknowledgements
	{\centering\Huge\bfseries\underline{Acknowledgments}\par}
	\newpage
	
	% empty page
	\mbox{}
	\newpage
	
	%abstract
	{\centering\Huge\bfseries\underline{Abstract}\par}
	\newpage	
	
	% insert the table of contents
	\tableofcontents
	\newpage
		
	% list of figures here
	\listoffigures
	\newpage
 
 	%list of tables here
	\listoftables
	\newpage
	
	% new section
	\pagenumbering{arabic}
	\newpage
	\pagenumbering{arabic}
	\newpage
	
	% introduction
\section{Introduction}
	Well, and here begins my lovely article.
	\subsection{Subject and motivation for the Research}
		meh
	\subsection{Background to the Research}
		adding it here
	\subsection{ Objectives of this Research}
		adding more
		\subsubsection{The Significance of the Research}
			a little here
	\subsection{Scope and Limitations of the Research}
		something something	
	\subsection{Plan of Development}	
		blah blah
	\newpage

	% literature review
\section{Literature Review}
	some more stuff
	\newpage
	
	% design and prototyping
\section{Design and Prototyping Methodology and Procedure}
	In order to begin the design process a clear methodology was needed to proceed in order to get the best results. This included a set of rules to follow when designing and testing prototypes and more. This section aims to discuss these and elaborate on why it will make the design process more effective. 

\subsection{Design}
The methodology behind the mechanical design will be reviewed first then circuit design, software design and and finally prototyping:
\subsubsection{Mechanical Design}
In order to make an effective design certain constraints were first laid out to limit the scope and complexity of the design:
\begin{itemize}
	\item The design should be simple to limit complex movements.
	\item Although it should be simple, simplicity should not be the main priority where complexity is needed.
	\item Reduce moving parts in order to limit mechanical failure.
	\item Identify failure points and modes.
	\item 	 
\end{itemize}

\subsubsection{Circuit Design Methodology}
The circuit design although trivial needs some consideration before design can begin:
\begin{itemize}
	\item 	 
\end{itemize}

\subsubsection{Software Design Methodology}
The software for the machine is one of the most important parts of consideration as it will impact each part of the design and how they interact, this and more must be considered when designing the software:
\begin{itemize}
	\item 	 
\end{itemize}

\subsubsection{Prototyping Methodology and Procedure }
Detailed planning and methodology was needed in order to test the viability of the prototypes for the final build:
\begin{itemize}
	\item 	 
\end{itemize}


	%the end
\end{document}
